% In this file you should put the actual content of the blueprint.
% It will be used both by the web and the print version.
% It should *not* include the \begin{document}
%
% If you want to split the blueprint content into several files then
% the current file can be a simple sequence of \input. Otherwise It
% can start with a \section or \chapter for instance.

\section{reflection groupsとroot lattices}
ここで登場する記号はMathlibとは関係ない独立したものであり,証明もMathlibのものとは異なる場合がある.

以下,$V$を$\R^n$のsubspaceとし,$\alpha, \beta \in V$の内積を$\inner{\alpha}{\beta}$と書き,$\alpha$のノルムを$\norm{\alpha} = \sqrt{\inner{\alpha}{\alpha}}$と書く.
\begin{defi}
  \label{defi:orthogonal_transformation}
  $f : \map{V}{V}$を線型写像とする.
  このとき,$f$が\textbf{直交変換}であるとは,
  \begin{equation}
    \inner{f(\alpha)}{f(\beta)} = \inner{\alpha}{\beta}
  \end{equation}
  が成り立つことである.
  また,$V$上の直交変換全体の集合を$O(V)$と書く.
\end{defi}

\begin{rem}
  \label{rem:orthogonal_transformation_norm}
  \lean{LinearIsometryEquiv}
  \leanok
  定義から$\norm{f(\alpha)} = \norm{\alpha}$がすぐわかる.
\end{rem}

\begin{rem}
  \label{rem:orthogonal_transformation_one-to-one}
  $f \in O(\R^n)$とする.
  $O(n) := \set{A}{\transpose{A} A = I_n}$ (i.e., 直交行列全体)とすると,$A \in O(n)$を用いて$f(x) = Ax$と表せる.
  よって,次の1対1対応がある:
  \begin{equation}
    \begin{array}{ccc}
    O(n) & \xleftrightarrow[\text{one-to-one}]{\simeq}  & O(\R^n) \\
    \rotatebox[origin=c]{90}{$\in$} & & \rotatebox[origin=c]{90}{$\in$} \\
    A & \longleftrightarrow & f(x) = Ax
    \end{array}
  \end{equation}
\end{rem}

以下,$\bm{k}$を$\R$または$\C$とし(Leanでは$\mathtt{RCLike}$),$E$を$\bm{k}$-内積空間,$K$を$E$の$\bm{k}$-部分加群とする.

\begin{defi}
  \label{defi:orthogonalProjection}
  \lean{Submodule.orthogonalProjection}
  \leanok
  $\forall v \in E,\ \exists w \in K \st v - w \in K^\perp$とする.
  線型連続写像$\operatorname{proj}_K : \map{E}{K}; \operatorname{proj}_K(v) = w$を\textbf{正射影(projection)}という.
  すなわち,$x \in E$を$x = \operatorname{proj}_K(x) + (x - \operatorname{proj}_K(x)) \in K \oplus K^\perp$と書ける.
\end{defi}

\begin{thm}
  \label{thm:orthogonalProjection_singleton}
  \lean{Submodule.orthogonalProjection_singleton}
  \leanok
  任意の$v, w \in E$に対して次が成り立つ:
  \begin{equation}
    \operatorname{proj}_{\bm{k}v}(w) = \frac{\inner{v}{w}}{\norm{v}^2} v.
  \end{equation}
\end{thm}

\begin{proof}
  \leanok
  略.
\end{proof}

\begin{defi}
  \label{defi:reflection}
  \lean{Submodule.reflection}
  \leanok
  次の直交変換を\textbf{reflection}という:
  \begin{equation}
    s_K : \map{E}{E};\ x \mapsto 2 \cdot \operatorname{proj}_K(x) - x.
  \end{equation}
\end{defi}

\begin{thm}
  \label{thm:reflection_singleton_apply}
  \lean{Submodule.reflection_singleton_apply}
  \leanok
  任意の$u, v \in E$に対し,次が成り立つ:
  \begin{equation}
    s_{\bm{k}u}(v) = 2 \frac{\inner{u}{v}}{\norm{u}^2}u - v.
  \end{equation}
\end{thm}

\begin{proof}
  \leanok
  \uses{thm:orthogonalProjection_singleton}
  略.
\end{proof}

\begin{rem}
  以前,$\alpha$に関するreflectionは
  \begin{equation}
    s_\alpha (\lambda) = \lambda - 2\frac{\inner{\alpha}{\lambda}}{\inner{\alpha}{\alpha}} \alpha.
  \end{equation}
  と表せることを見た.
  定理~\ref{thm:reflection_singleton_apply}からわかるように,$s_{\bm{k}\alpha}$は$s_\alpha$と逆向き(すなわち,$s_\alpha = s_{H_\alpha} = s_{(\R \alpha)^\perp}$)である.
\end{rem}

\begin{thm}
  \label{thm:reflection_eq_self_iff}
  \lean{Submodule.reflection_eq_self_iff}
  \leanok
  任意の$x \in E$に対して次が成り立つ:
  \begin{equation}
    s_K(x) = x \iff x \in K.
  \end{equation}
\end{thm}

\begin{proof}
  \leanok
  略.
\end{proof}

\begin{thm}
  \label{thm:reflection_map_apply}
  \lean{Submodule.reflection_map_apply}
  \leanok
  $E'$を$\bm{k}$-内積空間,$f : \map{E}{E'}$を線形同型な等長写像(すなわち,$\norm{f(x)} = \norm{x}$)とする.
  このとき,任意の$x \in E'$に対して次が成り立つ:
  \begin{equation}
    s_{f(K)}(x) = f(s_K(f^{-1}(x))).
  \end{equation}
\end{thm}

\begin{proof}
  \leanok
  略.
\end{proof}

\begin{thm}
  \label{thm:reflection_mem_subspace_orthogonalComplement_eq_neg}
  \lean{Submodule.reflection_mem_subspace_orthogonalComplement_eq_neg}
  \leanok
  任意の$v \in E$に対し,$v \in K^\perp$なら$s_K(v) = -v$である.
\end{thm}

\begin{proof}
  \leanok
  略.
\end{proof}

\begin{defi}
  $O(V)$の部分群$W$が次を満たすとき,$W$を\textbf{finite reflection group}という:
  \begin{enumerate}[label=(\roman*)]
    \item $W$はfinite group,
    \item $W$はreflectionsで生成される,すなわち
    \begin{equation}
      \forall w \in W,\ \exists s_{\alpha_1}, \ldots, s_{\alpha_r} \in W \textrm{: reflections} \st w = s_{\alpha_1} \cdots s_{\alpha_r}.
    \end{equation}
  \end{enumerate}
\end{defi}

\begin{rem}
  $s_\alpha^2 = 1.$
\end{rem}

\begin{defi}
  finite reflection group $W$が次を満たすとき,$W$は\textbf{essential}であるという:
  \begin{equation}
    \Fix(W) := \set{\lambda \in V}{\forall w \in W,\ w(\lambda) = \lambda} = \{0\}.
  \end{equation}
\end{defi}

\begin{rem}
  $W$がessentialでないとき,$V = \Fix(W) \oplus \Fix(W)^\perp$である.
  また,部分空間$\Fix(W)^\perp$上では$W$はessentialである.
\end{rem}

\begin{defi}
  空でない$V$の有限部分集合$\Phi$が次を満たすとき,$\Phi$を\textbf{root system}という:
  \begin{enumerate}[label=(\roman*)]
    \item $\Phi$は$V$を生成する, \label{defi:root_gen}
    \item $\forall \alpha \in \Phi,\ \R \alpha \cap \Phi = \{\pm \alpha\}$, \label{defi:root_self}
    \item $\forall \alpha, \beta \in \Phi,\ s_\beta(\alpha) \in \Phi$. \label{defi:root_ref_closed}
  \end{enumerate}
  また,$\Phi$の元を\textbf{root vector},または単に\textbf{root}という.
\end{defi}

\begin{defi}
  root system $\Phi$が次を満たすとき,\textbf{crystallographic}であるという:
  \begin{enumerate}[label=(\roman*)]
    \setcounter{enumi}{3}
    \item $\forall \alpha, \beta \in \Phi,\ 2 \dfrac{\inner{\alpha}{\beta}}{\inner{\alpha}{\alpha}} \in \Z$.
  \end{enumerate}
\end{defi}

finite reflection groupが与えられると,それに対応するroot systemが存在する:

\begin{prop} \label{prop:frg_to_root}
  $W$をessential finite reflection groupとする.
  このとき,$\Phi := \set{\alpha \in V}{\norm{\alpha} = 1,\ s_\alpha \in W}$はroot systemである.
\end{prop}
\begin{proof}
  \ref{defi:root_gen}を示すために,$\Fix (W) = \bigcap_{\alpha \in \Phi} H_\alpha$を示す:

  $(\subseteq)$任意に$\lambda \in \Fix(W),\ \alpha \in \Phi$をとると,$s_\alpha \in W$であるから$s_\alpha(\lambda) = \lambda$である.
  よって,
  \begin{equation}
    \inner{\alpha}{\lambda}
    = \inner{\alpha}{s_\alpha(\lambda)}
    = \inner{s_\alpha^{-1}(\alpha)}{\lambda}
    = \inner{s_\alpha(\alpha)}{\lambda}
    = \inner{-\alpha}{\lambda}
    = -\inner{\alpha}{\lambda}.
    \qquad \therefore \inner{\alpha}{\lambda} = 0.
  \end{equation}
  したがって,$\lambda \in H_\alpha$である.

  $(\supseteq)$ $\lambda \in \bigcap_{\alpha \in \Phi} H_\alpha$とする.
  任意に$w \in W$をとり,$w = s_{\alpha_1} \cdots s_{\alpha_r}$とreflectionsの積で書く.
  ただし,$\norm{\alpha_1} = \cdots = \norm{\alpha_r} = 1$としておく.
  すると,$\alpha_1, \ldots, \alpha_r \in \Phi$であるから,$\lambda \in H_{\alpha_i}$である.
  よって,
  \begin{equation}
    w(\lambda)
    = (s_{\alpha_1} \cdots s_{\alpha_{r-1}} s_{\alpha_r})(\lambda)
    = (s_{\alpha_1} \cdots s_{\alpha_{r-1}})(\lambda)
    = \cdots
    = \lambda.
  \end{equation}

  よって,$W$はessentialだから,
  \begin{equation}
    \{0\}
    = \Fix(W)
    = \bigcap_{\alpha \in \Phi} H_\alpha.
  \end{equation}
  したがって,
  \begin{equation}
    V
    = \Fix(W)^\perp
    = \left( \bigcap_{\alpha \in \Phi} H_\alpha \right)^\perp
    = \sum_{\alpha \in \Phi} H_\alpha^\perp
    = \sum_{\alpha \in \Phi} \R\alpha
  \end{equation}
  であるから$\Phi$は$V$を生成する.

  \ref{defi:root_self}は$\norm{\alpha} = 1$からわかる.

  \ref{defi:root_ref_closed}を示す.
  $\forall \alpha, \beta \in \Phi$に対し,$\norm{s_\beta(\alpha)} = \norm{\alpha} = 1$である.
  また,補題~\ref{lem:orth_ref_orthinv}より,$s_{s_\beta(\alpha)} = s_\beta s_\alpha s_\beta^{-1} \in W$である.
  よって,$s_\beta(\alpha) \in \Phi$が成り立つ.
\end{proof}

逆に,root systemが与えられると,それに対応するfinite reflection groupが存在する:
\begin{prop}
  $\Phi \subseteq V$をroot systemとする.
  このとき,$\set{s_\alpha}{\alpha \in \Phi}$が生成する群$W_\Phi$はessential finite reflection groupである.
\end{prop}
\begin{proof}
  命題~\ref{prop:frg_to_root}の証明中\ref{defi:root_gen}で,$\Phi = \{\alpha_1 \ldots, \alpha_r\}$とし,$W$を$W_\Phi$と読み替えれば$\Fix(W_\Phi) = \bigcap_{\alpha \in \Phi} H_\alpha$が成り立つ.
  よって,$\Phi$は$V$を生成するから$\Fix(W_\Phi)^\perp = \sum_{\alpha \in \Phi} \R\alpha = V$である.
  したがって,$W_\Phi$はessentialである.

  あとは$W_\Phi$が有限であることを示せば良い.
  \ref{defi:root_ref_closed}より,$\forall \alpha \in \Phi,\ s_\alpha(\Phi) = \Phi$であるから$\forall w \in W,\ w(\Phi) = \Phi$である.
  すなわち,$w$は$\Phi$上の置換とみなせる.
  よって,
  \begin{equation}
    \begin{array}{rccc}
    p:&W_\Phi & \longrightarrow & \operatorname{Perm}(\Phi) \\
    &\rotatebox[origin=c]{90}{$\in$} & & \rotatebox[origin=c]{90}{$\in$} \\
    &w & \longmapsto & (\alpha \mapsto w(\alpha))
    \end{array}
  \end{equation}
  と定めると,これはgroup homである($\operatorname{Perm}(\Phi)$は$\Phi$の置換群).
  これが単射であることを示せば,$\abs{W_\Phi} \le \abs{\operatorname{Perm}(\Phi)} < \infty$が示せる.
  \begin{align}
    w \in \Ker p
    &\iff \forall \alpha \in \Phi,\ w(\alpha) = \alpha\\
    &\iff w = 1
  \end{align}
  であるから,$\Ker p = \{1\}$であり,単射であることが示せた.
\end{proof}

root systemはfinite reflection groupの生成系を与えているが,群の生成元の個数はできるだけ少なくしたい.
そこで,root systemを``うまく''取る方法を考える.

\begin{defi}
  $\Phi \subseteq V$をroot systemとし,$p \in V \setminus \{0\}$は$\forall \alpha \in \Phi,\ \inner{\alpha}{p} \ne 0$を満たすとする.
  このとき,$\Pi := \set{\alpha \in \Phi}{\inner{\alpha}{p} > 0}$を\textbf{positive root system}という.
\end{defi}

\begin{rem}
  $\Pi$は$p$の取り方による.
\end{rem}

\begin{defi}
  $\Pi$をpositive root systemとする.
  $\Delta \subseteq \Pi$が次を満たすとき,$\Delta$を\textbf{simple root system}という:
  \begin{enumerate}[label=(\roman*)]
    \item $\Delta$は$V$の基底,
    \item $\forall \alpha \in \Pi,\ \exists c_\beta \ge 0 \st \alpha = \sum_{\beta \in \Delta} c_\beta \beta$.
  \end{enumerate}
\end{defi}

simple root systemは必ず存在し,しかも一意である:

\begin{fact}
  $\Phi$をroot system,$\Pi$をpositive root systemとする.
  \begin{enumerate}[label=(\arabic*)]
    \item $\exists! \Delta \subseteq \Phi$ : simple root system,
    \item $W_\Phi$は$\set{s_\alpha}{\alpha \in \Delta}$で生成される.
  \end{enumerate}
\end{fact}

以下,essential finite reflection group $W$に対し,そのroot systemを$\Phi$,そのsimple root systemを$\Delta$とする.
また,$\alpha, \beta \in \Delta$に対し,$m(\alpha, \beta)$を$s_\alpha s_\beta$の位数,$c(\beta, \alpha) = 2\dfrac{\inner{\beta}{\alpha}}{\inner{\alpha}{\alpha}}$とする.

\begin{rem}
  $s_\alpha^2 = 1$より$m(\alpha, \alpha) = 1$である.
  また,$s_\beta s_\alpha = (s_\alpha s_\beta)^{-1}$より$m(\alpha, \beta) = m(\beta, \alpha)$である.
\end{rem}

\begin{lem} \label{lem:inner_order}
  $\alpha \ne \beta \in \Delta$とする.このとき,次が成り立つ:
  \begin{equation}
    \inner{\alpha}{\beta}
    = -\norm{\alpha} \norm{\beta} \cos\frac{\pi}{m(\alpha, \beta)}.
  \end{equation}
\end{lem}

\begin{prop}
  $\alpha$と$\beta$は線型独立とする.$\Phi$がcrystallographicなとき,$m(\alpha, \beta) = 2, 3, 4, 6$である.
\end{prop}
\begin{proof}
  補題~\ref{lem:inner_order}より,
  \begin{equation}
    c(\beta, \alpha)
    = 2\frac{\inner{\beta}{\alpha}}{\inner{\alpha}{\alpha}}
    = 2\frac{-\norm{\alpha} \norm{\beta} \cos\frac{\pi}{m(\alpha, \beta)}}{\norm{\alpha}^2}
    = -2\frac{\norm{\beta}}{\norm{\alpha}} \cos\frac{\pi}{m(\alpha, \beta)}
  \end{equation}
  であるから,
  \begin{equation}
    c(\alpha, \beta) c(\beta, \alpha)
    = \left( -2\frac{\norm{\alpha}}{\norm{\beta}} \cos\frac{\pi}{m(\beta, \alpha)} \right) \left( -2\frac{\norm{\beta}}{\norm{\alpha}} \cos\frac{\pi}{m(\alpha, \beta)} \right)
    = 4\cos^2 \frac{\pi}{m(\alpha, \beta)}
  \end{equation}
  である.
  $\Phi$はcrystallographic,すなわち$c(\alpha, \beta), c(\beta, \alpha) \in \Z$であるから,$c(\alpha, \beta) c(\beta, \alpha) = 0, 1, 2, 3, 4$である.

  $c(\alpha, \beta) c(\beta, \alpha) = 4$のとき,$m(\alpha, \beta) = 1$となるが,これは$\alpha$と$\beta$の線型独立性に矛盾する.

  $c(\alpha, \beta) c(\beta, \alpha) = 0, 1, 2, 3$のとき,それぞれ次のようになる:
  \begin{table}[htbp]
    \centering
    \caption{$c(\alpha, \beta) c(\beta, \alpha)$と$m(\alpha, \beta)$の関係}
    \label{fig:c_m_relation}
    \begin{tabular}{cccc}
      $c(\alpha, \beta) c(\beta, \alpha)$ & $\cos^2 \frac{\pi}{m(\alpha, \beta)}$ & $\cos\frac{\pi}{m(\alpha, \beta)}$ & $m(\alpha, \beta)$\\ \hline
      $0$ & $0$ & $0$ & $2$ \\
      $1$ & $1/4$ & $1/2$ & $3$\\
      $2$ & $1/2$ & $1/\sqrt2$ & $4$\\
      $3$ & $3/4$ & $\sqrt{3}/2$ & $6$
    \end{tabular}
  \end{table}
\end{proof}

\begin{defi}
  $\Delta = \{\alpha_1, \ldots, \alpha_r\}$とする.
  このとき,$r$次正方行列$C := (c(\alpha_i, \alpha_j))$を$\Phi$の\textbf{Cartan matrix}という.
\end{defi}

\begin{defi}
  $\Delta = \{\alpha_1, \ldots, \alpha_r\}$とする.
  このとき,$\Phi$の\textbf{Coxeter diagram}を次のように定義する:
  \begin{enumerate}[label=(\roman*)]
    \item $r$個の頂点を持つ,
    \item 頂点$i$と頂点$j$は$m(\alpha_i, \alpha_j)$と書かれた辺で結ぶ,\label{defi:node_m}
    \item 特に,$m(\alpha_i, \alpha_j) = 2, 3, 4, 6$のときは,\ref{defi:node_m}の代わりに表~\ref{fig:c_m_relation}で対応する$c(\alpha_i, \alpha_j) c(\alpha_j, \alpha_i)$本の辺で結ぶ.
  \end{enumerate}
\end{defi}

\begin{defi}
  Coxeter diagramに次の条件を追加したものを\textbf{Coxeter-Dynkin diagram}という:
  \begin{enumerate}[label=(\roman*)]
    \setcounter{enumi}{3}
    \item $\abs{c(\alpha_i, \alpha_j)} < \abs{c(\alpha_j, \alpha_i)}$のとき,頂点$i$から頂点$j$に向きをつける.
  \end{enumerate}
\end{defi}

\begin{defi}
  root system $\Phi$で生成されるlattice,すなわち次の集合$\Lambda$を\textbf{root lattice}という:
  \begin{equation}
    \Lambda := \set{\sum_{i=1}^{r} s_i \alpha_i}{\alpha_i \in \Phi,\ s_i \in \Z}.
  \end{equation}
\end{defi}

\section{$E_8$ lattice}

\begin{defi}
  \label{defi:E8Lattice}
  \lean{E8Lattice}
  \leanok
  \uses{defi:IntegralLattice}
  $E_8$格子とは,integralLatticeであって,even unimodularかつランクが$8$であるもののことである.
\end{defi}

\begin{thm}
  \label{thm:unique}
  \lean{E8Lattice.unique}
  \leanok
  2つの$E_8$格子$\Lambda_1,\ \Lambda_2$は同型である.
\end{thm}

\begin{proof}
  sorry.
\end{proof}

\begin{defi}
  \label{defi:M0-M7}
  \lean{E8Lattice.M0}
  \leanok
  $E_8$のCartan行列を$M_0$,それを1行ずつ行基本変形していき(その過程の行列を$M_1, M_2, \ldots, M_6$とする)上三角にしたものを$M_7$とする:
  \begin{gather}
    M_0 :=
    \begin{pmatrix}
      2 & 0 & -1 & 0 & 0 & 0 & 0 & 0 \\
      0 & 2 & 0 & -1 & 0 & 0 & 0 & 0 \\
      -1 & 0 & 2 & -1 & 0 & 0 & 0 & 0 \\
      0 & -1 & -1 & 2 & -1 & 0 & 0 & 0 \\
      0 & 0 & 0 & -1 & 2 & -1 & 0 & 0 \\
      0 & 0 & 0 & 0 & -1 & 2 & -1 & 0 \\
      0 & 0 & 0 & 0 & 0 & -1 & 2 & -1 \\
      0 & 0 & 0 & 0 & 0 & 0 & -1 & 2
    \end{pmatrix},\\
    M_7 :=
    \begin{pmatrix}
      2 & 0 & -1 & 0 & 0 & 0 & 0 & 0 \\
      0 & 2 & 0 & -1 & 0 & 0 & 0 & 0 \\
      0 & 0 & 3/2 & -1 & 0 & 0 & 0 & 0 \\
      0 & 0 & 0 & 5/6 & -1 & 0 & 0 & 0 \\
      0 & 0 & 0 & 0 & 4/5 & -1 & 0 & 0 \\
      0 & 0 & 0 & 0 & 0 & 3/4 & -1 & 0 \\
      0 & 0 & 0 & 0 & 0 & 0 & 2/3 & -1 \\
      0 & 0 & 0 & 0 & 0 & 0 & 0 & 1/2
    \end{pmatrix}.
  \end{gather}
\end{defi}

\begin{lem}
  \label{lem:M7_upperTrianglar}
  \lean{E8Lattice.M7_upperTrianglar}
  \leanok
  $M_7$は上三角である.
\end{lem}

\begin{proof}
  \leanok
  \uses{defi:M0-M7}
  略.
\end{proof}

\begin{lem}
  \label{lem:M7_det}
  \lean{E8Lattice.M7_det}
  \leanok
  $\det M_7 = 1$である.
\end{lem}

\begin{proof}
  \leanok
  \uses{lem:M7_upperTrianglar, defi:M0-M7}
  補題~\ref{lem:M7_upperTrianglar}より,$M_7$の行列式は対角成分たちの積であるから
  \begin{equation}
    \det M_7
    = 2 \cdot 2 \cdot (3 / 2) \cdot (5 / 6) \cdot (4 / 5) \cdot (3 / 4) \cdot (2 / 3) \cdot (1 / 2)\\
    = 1.
  \end{equation}
\end{proof}

\begin{thm}
  \label{thm:E8_det}
  \lean{E8Lattice.E8_det}
  \leanok
  $E_8$のCartan行列の行列式は$1$である.
\end{thm}

\begin{proof}
  \uses{defi:M0-M7, lem:M7_det}
  補題~\ref{lem:M7_det}より
  \begin{equation}
    (\textrm{求める行列式})
    = \det M_0
    = \det M_1
    = \cdots
    = \det M_7
    = 1.
  \end{equation}
\end{proof}

\begin{defi}
  \label{defi:B}
  \lean{E8Lattice.B}
  \leanok
  $B$を$E_8$のCartan行列$C (= M_0) \in \M_8(\Z)$から定まる双線型形式とする:
  \begin{equation}
    B(x, y) := \transpose{x} C y = \inner{x}{Cy} \qquad (\forall x, y \in \Z^8).
  \end{equation}
\end{defi}

\begin{lem}
  \label{lem:inner_self_calc}
  \lean{E8Lattice.inner_self_calc}
  \leanok
  任意の$x \in \Z^8$に対し,次が成り立つ:
  \begin{equation}
    \begin{split}
      B(x, x) ={}& 2 (x_0^2 + x_1^2 + x_2^2 + x_3^2 + x_4^2 + x_5^2 + x_6^2 + x_7^2\\
      &- (x_0 x_2 + x_1 x_3 + x_2 x_3 + x_3 x_4 + x_4 x_5 + x_5 x_6 + x_6 x_7))
    \end{split}
  \end{equation}
\end{lem}

\begin{proof}
  \leanok
  \uses{defi:B}
  内積の形にして,あとは具体的に計算:
  \begin{equation}
    B(x, x)
    = \inner{x}{Cx}
    = (\textrm{右辺}).
  \end{equation}
\end{proof}

\begin{lem}
  \label{lem:inner_self_comp_sq}
  \lean{E8Lattice.inner_self_comp_sq}
  \leanok
  任意の$x \in \Z^8$に対し,平方完成すると次のようになる:
  \begin{equation}
    \begin{split}
      B(x, x) ={}& \left( \sqrt{2} \, x_0 - \sqrt{\frac{1}{2}} \, x_2 \right)^2 + \left( \sqrt{2} \, x_1 - \sqrt{\frac{1}{2}} \, x_3 \right)^2 + \left( \sqrt{\frac{3}{2}} \, x_2 - \sqrt{\frac{2}{3}} \, x_3 \right)^2 \\
      & + \left( \sqrt{\frac{5}{6}} \, x_3 - \sqrt{\frac{6}{5}} \, x_4 \right)^2 + \left( \sqrt{\frac{4}{5}} \, x_4 - \sqrt{\frac{5}{4}} \, x_5 \right)^2 + \left( \sqrt{\frac{3}{4}} \, x_5 - \sqrt{\frac{4}{3}} \, x_6 \right)^2 \\
      & + \left( \sqrt{\frac{2}{3}} \, x_6 - \sqrt{\frac{3}{2}} \, x_7 \right)^2 + \frac{1}{2} \, x_7^2
    \end{split}
  \end{equation}
\end{lem}

\begin{proof}
  \leanok
  \uses{lem:inner_self_calc}
  左辺に補題~\ref{lem:inner_self_calc}を代入して計算すれば得られる.
\end{proof}

\begin{thm}
  \label{thm:add_inner}
  \lean{E8Lattice.AddInner}
  \leanok
  $\forall x, y, z \in \Z^8,\ B(x+y, z) = B(x, z) + B(y, z)$.
\end{thm}

\begin{proof}
  \leanok
  \uses{defi:B}
  計算するだけ.
\end{proof}

\begin{thm}
  \label{thm:inner_sym}
  \lean{E8Lattice.InnerSym}
  \leanok
  $\forall x, y \in \Z^8,\ B(x, y) = B(y, x)$.
\end{thm}

\begin{proof}
  \leanok
  \uses{defi:B}
  計算するだけ.
\end{proof}

\begin{thm}
  \label{thm:inner_self}
  \lean{E8Lattice.InnerSelf}
  \leanok
  $\forall x \in \Z^8,\ B(x, x) \ge 0$.
\end{thm}

\begin{proof}
  \leanok
  \uses{lem:inner_self_comp_sq}
  補題~\ref{lem:inner_self_comp_sq}より,$B(x, x)$は平方の和で表せるから成り立つ.
\end{proof}

\begin{thm}
  \label{thm:inner_self_eq_zero}
  \lean{E8Lattice.InnerSelfEqZero}
  \leanok
  $\forall x \in \Z^8,\ B(x, x) = 0 \implies x = 0$.
\end{thm}

\begin{proof}
  \leanok
  \uses{lem:inner_self_comp_sq}
  補題~\ref{lem:inner_self_comp_sq}より,$B(x, x)$は平方の和で表せ,$= 0$とすると各項が$0$である.
  よって,最後の項に注目すると,$x_7 = 0$である.
  したがって,最後から2番目の項に注目すると,$x_6 = 0$である.
  これを繰り返すと,$x_0 = \cdots = x_7 = 0$を得る.
\end{proof}

\begin{thm}
  \label{thm:even}
  \lean{E8Lattice.InnerEven}
  \leanok
  $\forall x \in \Z^8,\ 2 \mid \inner{x}{x}_{\Z}$.
\end{thm}

\begin{proof}
  \leanok
  \uses{lem:inner_self_calc}
  補題~\ref{lem:inner_self_calc}より従う.
\end{proof}

\begin{thm}
  \label{thm:unimodular}
  %\lean{E8Lattice.unimodular}
  \leanok
  $E_8$格子はunimodularである.
\end{thm}

\begin{proof}
  sorry.
\end{proof}

\begin{thm}
  \label{thm:exists_E8}
  \lean{E8Lattice.exists_E8}
  \leanok
  $E_8$格子は存在する.
\end{thm}

\begin{proof}
  sorry.
\end{proof}

\begin{thm}
  $E_8$格子$\Lambda$に対し,$\forall n \in \N,\ \#\set{x \in \Lambda}{B(x, x) = n} < \infty$.
\end{thm}

\begin{proof}
  sorry.
\end{proof}

\begin{lem}
  \label{lem:card_norm_2}
  \lean{E8Lattice.card_norm_2}
  \leanok
  $E_8$格子$\Lambda$に対し,$\#\set{x \in \Lambda}{B(x, x) = 2} = 240$.
\end{lem}

\begin{proof}
  sorry.
\end{proof}
