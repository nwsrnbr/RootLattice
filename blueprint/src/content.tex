% In this file you should put the actual content of the blueprint.
% It will be used both by the web and the print version.
% It should *not* include the \begin{document}
%
% If you want to split the blueprint content into several files then
% the current file can be a simple sequence of \input. Otherwise It
% can start with a \section or \chapter for instance.

\section{Reflections}
以下,$\Bbbk$を$\R$または$\C$とし(Leanでは$\mathtt{RCLike}$),$E$を$\Bbbk$-内積空間,$K$を$E$の$\Bbbk$-部分加群とする.

\begin{definition}
  \label{defi:orthogonalProjection}
  \lean{Submodule.orthogonalProjection}
  \leanok
  $\forall v \in E,\ \exists w \in K \st v - w \in K^\perp$とする.
  線型連続写像$\proj_K : \map{E}{K}; \proj_K(v) = w$を\textbf{正射影(projection)}という.
  すなわち,$V = \proj_K(v) + (v - \proj_K(v)) \in K \oplus K^\perp$と書ける.
\end{definition}

\begin{theorem}
  \label{thm:orthogonalProjection_singleton}
  \lean{Submodule.orthogonalProjection_singleton}
  \leanok
  任意の$v, w \in E$に対して次が成り立つ:
  \begin{equation}
    \proj_{\Bbbk v}(w) = \frac{\inner{v}{w}}{\norm{v}^2} v.
  \end{equation}
\end{theorem}

\begin{proof}
  \leanok
  略.
\end{proof}

\begin{definition}
  \label{defi:reflection}
  \lean{Submodule.reflection}
  \leanok
  次の直交変換を\textbf{reflection}という:
  \begin{equation}
    s_K : \map{E}{E};\ x \mapsto 2 \cdot \proj_K(x) - x.
  \end{equation}
\end{definition}

\begin{proof}
  \leanok
  略.
\end{proof}

\begin{theorem}
  \label{thm:reflection_singleton_apply}
  \lean{Submodule.reflection_singleton_apply}
  \leanok
  任意の$u, v \in E$に対し,次が成り立つ:
  \begin{equation}
    s_{\Bbbk u}(v) = 2 \frac{\inner{u}{v}}{\norm{u}^2}u - v.
  \end{equation}
\end{theorem}

\begin{proof}
  \leanok
  \uses{thm:orthogonalProjection_singleton}
  略.
\end{proof}

\begin{remark}
  以前,$u$に関するreflectionは
  \begin{equation}
    s_u (v) = v - 2\frac{\inner{u}{v}}{\inner{u}{u}} u
  \end{equation}
  と表せることを見た.
  定理~\ref{thm:reflection_singleton_apply}からわかるように,$s_{\Bbbk \alpha}$は$s_\alpha$と逆向きである.
\end{remark}

\begin{theorem}
  \label{thm:reflection_eq_self_iff}
  \lean{Submodule.reflection_eq_self_iff}
  \leanok
  任意の$x \in E$に対して次が成り立つ:
  \begin{equation}
    s_K(x) = x \iff x \in K.
  \end{equation}
\end{theorem}

\begin{proof}
  \leanok
  略.
\end{proof}

\begin{theorem}
  \label{thm:reflection_map_apply}
  \lean{Submodule.reflection_map_apply}
  \leanok
  $E'$を$\Bbbk $-内積空間,$f : \map{E}{E'}$を線形同型な等長写像(すなわち,$\norm{f(x)} = \norm{x}$)とする.
  このとき,任意の$x \in E'$に対して次が成り立つ:
  \begin{equation}
    s_{f(K)}(x) = f(s_K(f^{-1}(x))).
  \end{equation}
\end{theorem}

\begin{proof}
  \leanok
  略.
\end{proof}

\begin{theorem}
  \label{thm:reflection_mem_subspace_orthogonalComplement_eq_neg}
  \lean{Submodule.reflection_mem_subspace_orthogonalComplement_eq_neg}
  \leanok
  任意の$v \in E$に対し,$v \in K^\perp$なら$s_K(v) = -v$である.
\end{theorem}

\begin{proof}
  \leanok
  略.
\end{proof}

\section{Reflection groupsとroot systems}
\begin{definition}
  \label{defi:FiniteReflectionGroup}
  \lean{FiniteReflectionGroup}
  \leanok
  群$W$が次を満たすとき,$W$を\textbf{finite reflection group}という:
  \begin{enumerate}[label=(\roman*)]
    \item $W$はfinite group,
    \item $W$はreflectionsで生成される,すなわち
    \begin{equation}
      \forall w \in W,\ \exists s_{\alpha_1}, \ldots, s_{\alpha_r} \in W \textrm{: reflections} \st w = s_{\alpha_1} \cdots s_{\alpha_r}.
    \end{equation}
  \end{enumerate}
\end{definition}

\begin{definition}
  finite reflection group $W$が次を満たすとき,$W$は\textbf{essential}であるという:
  \begin{equation}
    \Fix(W) := \set{\lambda \in V}{\forall w \in W,\ w(\lambda) = \lambda} = \{0\}.
  \end{equation}
\end{definition}

\begin{remark}
  $W$がessentialでないとき,$V = \Fix(W) \oplus \Fix(W)^\perp$である.
  また,部分空間$\Fix(W)^\perp$上では$W$はessentialである.
\end{remark}

\begin{definition}
  空でない$V$の有限部分集合$\Phi$が次を満たすとき,$\Phi$を\textbf{root system}という:
  \begin{enumerate}[label=(\roman*)]
    \item $\Phi$は$V$を生成する, \label{defi:root_gen}
    \item $\forall \alpha \in \Phi,\ \R \alpha \cap \Phi = \{\pm \alpha\}$, \label{defi:root_self}
    \item $\forall \alpha, \beta \in \Phi,\ s_\beta(\alpha) \in \Phi$. \label{defi:root_ref_closed}
  \end{enumerate}
  また,$\Phi$の元を\textbf{root vector},または単に\textbf{root}という.
\end{definition}

\begin{definition}
  root system $\Phi$が次を満たすとき,\textbf{crystallographic}であるという:
  \begin{enumerate}[label=(\roman*)]
    \setcounter{enumi}{3}
    \item $\forall \alpha, \beta \in \Phi,\ 2 \dfrac{\inner{\alpha}{\beta}}{\inner{\alpha}{\alpha}} \in \Z$.
  \end{enumerate}
\end{definition}

finite reflection groupが与えられると,それに対応するroot systemが存在する:

\begin{proposition} \label{prop:frg_to_root}
  $W$をessential finite reflection groupとする.
  このとき,$\Phi := \set{\alpha \in V}{\norm{\alpha} = 1,\ s_\alpha \in W}$はroot systemである.
\end{proposition}
\begin{proof}
  \ref{defi:root_gen}を示すために,$\Fix (W) = \bigcap_{\alpha \in \Phi} H_\alpha$を示す:

  $(\subseteq)$任意に$\lambda \in \Fix(W),\ \alpha \in \Phi$をとると,$s_\alpha \in W$であるから$s_\alpha(\lambda) = \lambda$である.
  よって,
  \begin{equation}
    \inner{\alpha}{\lambda}
    = \inner{\alpha}{s_\alpha(\lambda)}
    = \inner{s_\alpha^{-1}(\alpha)}{\lambda}
    = \inner{s_\alpha(\alpha)}{\lambda}
    = \inner{-\alpha}{\lambda}
    = -\inner{\alpha}{\lambda}.
    \qquad \therefore \inner{\alpha}{\lambda} = 0.
  \end{equation}
  したがって,$\lambda \in H_\alpha$である.

  $(\supseteq)$ $\lambda \in \bigcap_{\alpha \in \Phi} H_\alpha$とする.
  任意に$w \in W$をとり,$w = s_{\alpha_1} \cdots s_{\alpha_r}$とreflectionsの積で書く.
  ただし,$\norm{\alpha_1} = \cdots = \norm{\alpha_r} = 1$としておく.
  すると,$\alpha_1, \ldots, \alpha_r \in \Phi$であるから,$\lambda \in H_{\alpha_i}$である.
  よって,
  \begin{equation}
    w(\lambda)
    = (s_{\alpha_1} \cdots s_{\alpha_{r-1}} s_{\alpha_r})(\lambda)
    = (s_{\alpha_1} \cdots s_{\alpha_{r-1}})(\lambda)
    = \cdots
    = \lambda.
  \end{equation}

  よって,$W$はessentialだから,
  \begin{equation}
    \{0\}
    = \Fix(W)
    = \bigcap_{\alpha \in \Phi} H_\alpha.
  \end{equation}
  したがって,
  \begin{equation}
    V
    = \Fix(W)^\perp
    = \left( \bigcap_{\alpha \in \Phi} H_\alpha \right)^\perp
    = \sum_{\alpha \in \Phi} H_\alpha^\perp
    = \sum_{\alpha \in \Phi} \R\alpha
  \end{equation}
  であるから$\Phi$は$V$を生成する.

  \ref{defi:root_self}は$\norm{\alpha} = 1$からわかる.

  \ref{defi:root_ref_closed}を示す.
  $\forall \alpha, \beta \in \Phi$に対し,$\norm{s_\beta(\alpha)} = \norm{\alpha} = 1$である.
  また,補題~\ref{lem:orth_ref_orthinv}より,$s_{s_\beta(\alpha)} = s_\beta s_\alpha s_\beta^{-1} \in W$である.
  よって,$s_\beta(\alpha) \in \Phi$が成り立つ.
\end{proof}

逆に,root systemが与えられると,それに対応するfinite reflection groupが存在する:
\begin{proposition}
  $\Phi \subseteq V$をroot systemとする.
  このとき,$\set{s_\alpha}{\alpha \in \Phi}$が生成する群$W_\Phi$はessential finite reflection groupである.
\end{proposition}
\begin{proof}
  命題~\ref{prop:frg_to_root}の証明中\ref{defi:root_gen}で,$\Phi = \{\alpha_1 \ldots, \alpha_r\}$とし,$W$を$W_\Phi$と読み替えれば$\Fix(W_\Phi) = \bigcap_{\alpha \in \Phi} H_\alpha$が成り立つ.
  よって,$\Phi$は$V$を生成するから$\Fix(W_\Phi)^\perp = \sum_{\alpha \in \Phi} \R\alpha = V$である.
  したがって,$W_\Phi$はessentialである.

  あとは$W_\Phi$が有限であることを示せば良い.
  \ref{defi:root_ref_closed}より,$\forall \alpha \in \Phi,\ s_\alpha(\Phi) = \Phi$であるから$\forall w \in W,\ w(\Phi) = \Phi$である.
  すなわち,$w$は$\Phi$上の置換とみなせる.
  よって,
  \begin{equation}
    \begin{array}{rccc}
    p:&W_\Phi & \longrightarrow & \operatorname{Perm}(\Phi) \\
    &\rotatebox[origin=c]{90}{$\in$} & & \rotatebox[origin=c]{90}{$\in$} \\
    &w & \longmapsto & (\alpha \mapsto w(\alpha))
    \end{array}
  \end{equation}
  と定めると,これはgroup homである($\operatorname{Perm}(\Phi)$は$\Phi$の置換群).
  これが単射であることを示せば,$\abs{W_\Phi} \le \abs{\operatorname{Perm}(\Phi)} < \infty$が示せる.
  \begin{align}
    w \in \Ker p
    &\iff \forall \alpha \in \Phi,\ w(\alpha) = \alpha\\
    &\iff w = 1
  \end{align}
  であるから,$\Ker p = \{1\}$であり,単射であることが示せた.
\end{proof}

root systemはfinite reflection groupの生成系を与えているが,群の生成元の個数はできるだけ少なくしたい.
そこで,root systemを``うまく''取る方法を考える.

\section{Positive root systemsとsimple root systems}
\begin{definition}
  $\Phi \subseteq V$をroot systemとし,$p \in V \setminus \{0\}$は$\forall \alpha \in \Phi,\ \inner{\alpha}{p} \ne 0$を満たすとする.
  このとき,$\Pi := \set{\alpha \in \Phi}{\inner{\alpha}{p} > 0}$を\textbf{positive root system}という.
\end{definition}

\begin{remark}
  $\Pi$は$p$の取り方による.
\end{remark}

\begin{definition}
  $\Pi$をpositive root systemとする.
  $\Delta \subseteq \Pi$が次を満たすとき,$\Delta$を\textbf{simple root system}という:
  \begin{enumerate}[label=(\roman*)]
    \item $\Delta$は$V$の基底,
    \item $\forall \alpha \in \Pi,\ \exists c_\beta \ge 0 \st \alpha = \sum_{\beta \in \Delta} c_\beta \beta$.
  \end{enumerate}
\end{definition}

simple root systemは必ず存在し,しかも一意である:

\begin{fact}
  $\Phi$をroot system,$\Pi$をpositive root systemとする.
  \begin{enumerate}[label=(\arabic*)]
    \item $\exists! \Delta \subseteq \Phi$ : simple root system,
    \item $W_\Phi$は$\set{s_\alpha}{\alpha \in \Delta}$で生成される.
  \end{enumerate}
\end{fact}

\section{Finite reflection groupsの分類}
以下,essential finite reflection group $W$に対し,そのroot systemを$\Phi$,そのsimple root systemを$\Delta$とする.
また,$\alpha, \beta \in \Delta$に対し,$m(\alpha, \beta)$を$s_\alpha s_\beta$の位数,$c(\beta, \alpha) = 2\dfrac{\inner{\beta}{\alpha}}{\inner{\alpha}{\alpha}}$とする.

\begin{remark}
  $s_\alpha^2 = 1$より$m(\alpha, \alpha) = 1$である.
  また,$s_\beta s_\alpha = (s_\alpha s_\beta)^{-1}$より$m(\alpha, \beta) = m(\beta, \alpha)$である.
\end{remark}


\begin{lemma}
  \label{lem:inner_order}
  $\alpha \ne \beta \in \Delta$とする.このとき,次が成り立つ:
  \begin{equation}
    \inner{\alpha}{\beta}
    = -\norm{\alpha} \norm{\beta} \cos\frac{\pi}{m(\alpha, \beta)}.
  \end{equation}
\end{lemma}

\begin{proposition}
  $\alpha$と$\beta$は線型独立とする.$\Phi$がcrystallographicなとき,$m(\alpha, \beta) = 2, 3, 4, 6$である.
\end{proposition}
\begin{proof}
  補題~\ref{lem:inner_order}より,
  \begin{equation}
    c(\beta, \alpha)
    = 2\frac{\inner{\beta}{\alpha}}{\inner{\alpha}{\alpha}}
    = 2\frac{-\norm{\alpha} \norm{\beta} \cos\frac{\pi}{m(\alpha, \beta)}}{\norm{\alpha}^2}
    = -2\frac{\norm{\beta}}{\norm{\alpha}} \cos\frac{\pi}{m(\alpha, \beta)}
  \end{equation}
  であるから,
  \begin{equation}
    c(\alpha, \beta) c(\beta, \alpha)
    = \left( -2\frac{\norm{\alpha}}{\norm{\beta}} \cos\frac{\pi}{m(\beta, \alpha)} \right) \left( -2\frac{\norm{\beta}}{\norm{\alpha}} \cos\frac{\pi}{m(\alpha, \beta)} \right)
    = 4\cos^2 \frac{\pi}{m(\alpha, \beta)}
  \end{equation}
  である.
  $\Phi$はcrystallographic,すなわち$c(\alpha, \beta), c(\beta, \alpha) \in \Z$であるから,$c(\alpha, \beta) c(\beta, \alpha) = 0, 1, 2, 3, 4$である.

  $c(\alpha, \beta) c(\beta, \alpha) = 4$のとき,$m(\alpha, \beta) = 1$となるが,これは$\alpha$と$\beta$の線型独立性に矛盾する.

  $c(\alpha, \beta) c(\beta, \alpha) = 0, 1, 2, 3$のとき,それぞれ次のようになる:
  \begin{table}[htbp]
    \centering
    \caption{$c(\alpha, \beta) c(\beta, \alpha)$と$m(\alpha, \beta)$の関係}
    \label{fig:c_m_relation}
    \begin{tabular}{cccc}
      $c(\alpha, \beta) c(\beta, \alpha)$ & $\cos^2 \frac{\pi}{m(\alpha, \beta)}$ & $\cos\frac{\pi}{m(\alpha, \beta)}$ & $m(\alpha, \beta)$\\ \hline
      $0$ & $0$ & $0$ & $2$ \\
      $1$ & $1/4$ & $1/2$ & $3$\\
      $2$ & $1/2$ & $1/\sqrt2$ & $4$\\
      $3$ & $3/4$ & $\sqrt{3}/2$ & $6$
    \end{tabular}
  \end{table}
\end{proof}

\begin{definition}
  $\Delta = \{\alpha_1, \ldots, \alpha_r\}$とする.
  このとき,$r$次正方行列$C := (c(\alpha_i, \alpha_j))$を$\Phi$の\textbf{Cartan matrix}という.
  また,$r$次正方行列$M := (m(\alpha_i, \alpha_j))$を$\Phi$の\textbf{Coxeter matrix}という.
\end{definition}

Coxeter matrixの定義から分かるように,Coxeter matrixとCoxeter-Dynkin diagramは一対一に対応する.

\begin{example}
  後述の図~\ref{fig:coxeter_diag_classification}において,$E_8$のCertan matrix, Coxeter matrixはそれぞれ次のようになる:
  \begin{equation}
    C_{E_8} =
    \begin{pmatrix}
      2 & 0 & -1 & 0 & 0 & 0 & 0 & 0 \\
      0 & 2 & 0 & -1 & 0 & 0 & 0 & 0 \\
      -1 & 0 & 2 & -1 & 0 & 0 & 0 & 0 \\
      0 & -1 & -1 & 2 & -1 & 0 & 0 & 0 \\
      0 & 0 & 0 & -1 & 2 & -1 & 0 & 0 \\
      0 & 0 & 0 & 0 & -1 & 2 & -1 & 0 \\
      0 & 0 & 0 & 0 & 0 & -1 & 2 & -1 \\
      0 & 0 & 0 & 0 & 0 & 0 & -1 & 2
    \end{pmatrix},\qquad
    M_{E_8} =
    \begin{pmatrix}
      1 & 2 & 3 & 2 & 2 & 2 & 2 & 2 \\
      2 & 1 & 2 & 3 & 2 & 2 & 2 & 2 \\
      3 & 2 & 1 & 3 & 2 & 2 & 2 & 2 \\
      2 & 3 & 3 & 1 & 3 & 2 & 2 & 2 \\
      2 & 2 & 2 & 3 & 1 & 3 & 2 & 2 \\
      2 & 2 & 2 & 2 & 3 & 1 & 3 & 2 \\
      2 & 2 & 2 & 2 & 2 & 3 & 1 & 3 \\
      2 & 2 & 2 & 2 & 2 & 2 & 3 & 1
    \end{pmatrix}
  \end{equation}
\end{example}

\begin{definition}
  $\Delta = \{\alpha_1, \ldots, \alpha_r\}$とする.
  このとき,$\Phi$の\textbf{Coxeter diagram}を次のように定義する:
  \begin{enumerate}[label=(\roman*)]
    \item $r$個の頂点を持つ,
    \item 頂点$\alpha_i$と頂点$\alpha_j$は$m(\alpha_i, \alpha_j)$と書かれた辺で結ぶ,\label{defi:node_m}
    \item 特に,$m(\alpha_i, \alpha_j) = 2, 3, 4, 6$のときは,\ref{defi:node_m}の代わりに表~\ref{fig:c_m_relation}で対応する$c(\alpha_i, \alpha_j) c(\alpha_j, \alpha_i)$本の辺で結ぶ.
  \end{enumerate}
\end{definition}

\begin{definition}
  Coxeter diagramに次の条件を追加したものを\textbf{Coxeter-Dynkin diagram}という:
  \begin{enumerate}[label=(\roman*)]
    \setcounter{enumi}{3}
    \item $\abs{c(\alpha_i, \alpha_j)} < \abs{c(\alpha_j, \alpha_i)}$のとき,頂点$\alpha_i$から頂点$\alpha_j$に向きをつける.
  \end{enumerate}
\end{definition}

\begin{proposition}
  \label{prop:direct_prod_is_FRG}
  $V_1 \subseteq \R^m,\ V_2 \subseteq \R^n$とし,$W_1, W_2$をそれぞれ$V_1, V_2$上のfinite reflection groupsとする.
  このとき,$W_1 \times W_2$は$V_1 \oplus V_2 \subseteq \R^m \oplus \R^n = \R^{m+n}$上のfinite reflection groupである.
  ここで,
  \begin{equation}
    (w_1, w_2)(\lambda_1, \lambda_2) = (w_1(\lambda_1), w_2(\lambda_2)),\quad (w_1, w_2) \in W_1 \times W_2,\ (\lambda_1, \lambda_2) \in V_1 \oplus V_2
  \end{equation}
  と定める.
\end{proposition}

\begin{definition}
  \label{defi:irreducible}
  $W$をfinite refrection group,$G$をそれに対応するCoxeter-Dynkin diagramとする.
  このとき,$G$が連結な(すべての頂点が結ばれている)とき,$W$や$G$は\textbf{既約}という.
  Coxeter matrixの言葉でいえば,任意の成分が$0$でないときである.
\end{definition}

\begin{theorem}
  \label{thm:diag_is_simeq_implies_FRG_is_simeq}
  $\Phi, \Phi'$をroot systemsとし,これらに対応するsimple root systemsをそれぞれ$\Delta, \Delta'$とする.
  また,$G_\Delta, G_{\Delta'}$をそれぞれ$\Delta, \Delta'$のCoxeter-Dynkin diagramとする.
  このとき,$G_\Delta \simeq G_{\Delta'}$なら$W_\Phi \simeq W_{\Phi'}$である.
\end{theorem}

\begin{theorem}
  \label{thm:coxeter_diag_classification}
  既約essential finite reflection groupのsimple root systemから定まるCoxeter-Dynkin diagramは,次の図においてcrystallographicであれば上9つ,そうでなければ下3つのいずれかである:
\end{theorem}

%\iffalse
\newpage
  \begin{figure}[b]
    \begin{center}
      \begin{tikzpicture}[scale=1.0]
        % A_n
        \foreach \i in {1,...,7}
					\node[circle, draw, fill=white, inner sep=1.5pt] (\i) at (\i,0) {};
				\foreach \i in {1,...,4,6}
					\draw (\i) -- (\the\numexpr\i+1\relax);
				\draw[dashed] (5) -- (6);
        \node[below] at (1) {$\alpha_1$};
        \node[below] at (2) {$\alpha_2$};
        \node[below] at (3) {$\alpha_3$};
        \node[below] at (4) {$\alpha_4$};
        \node[below] at (5) {$\alpha_5$};
        \node[below] at (6) {$\alpha_{n-1}$};
        \node[below] at (7) {$\alpha_n$};
        \node[left] at (1) {$A_n$ : \mbox{\hspace{10pt}}};
        \node[right] at (7) {\mbox{\hspace{10pt}}$(n \ge 1)$};

        % B_n
        \foreach \i in {1,...,7}
					\node[circle, draw, fill=white, inner sep=1.5pt] (\i) at (\i,-1.5) {};
				\foreach \i in {2,...,4,6}
					\draw (\i) -- (\the\numexpr\i+1\relax);
				\draw[double distance=3pt, postaction={decorate}, decoration={markings,mark=at position 0.7 with {\arrow{>}}}] (2) -- (1);
				\draw[dashed] (5) -- (6);
        \node[below] at (1) {$\alpha_1$};
        \node[below] at (2) {$\alpha_2$};
        \node[below] at (3) {$\alpha_3$};
        \node[below] at (4) {$\alpha_4$};
        \node[below] at (5) {$\alpha_5$};
        \node[below] at (6) {$\alpha_{n-1}$};
        \node[below] at (7) {$\alpha_n$};
        \node[left] at (1) {$B_n$ : \mbox{\hspace{10pt}}};
        \node[right] at (7) {\mbox{\hspace{10pt}}$(n \ge 2)$};

        % C_n
				\foreach \i in {1,...,7}
					\node[circle, draw, fill=white, inner sep=1.5pt] (\i) at (\i,-3) {};
				\foreach \i in {2,...,4,6}
					\draw (\i) -- (\the\numexpr\i+1\relax);
				\draw[double distance=3pt, postaction={decorate}, decoration={markings,mark=at position 0.7 with {\arrow{>}}}] (1) -- (2);
				\draw[dashed] (5) -- (6);
        \node[below] at (1) {$\alpha_1$};
        \node[below] at (2) {$\alpha_2$};
        \node[below] at (3) {$\alpha_3$};
        \node[below] at (4) {$\alpha_4$};
        \node[below] at (5) {$\alpha_5$};
        \node[below] at (6) {$\alpha_{n-1}$};
        \node[below] at (7) {$\alpha_n$};
        \node[left] at (1) {$C_n$ : \mbox{\hspace{10pt}}};
        \node[right] at (7) {\mbox{\hspace{10pt}}$(n \ge 2)$};

        % D_n
        \foreach \i in {2,...,7}
					\node[circle, draw, fill=white, inner sep=1.5pt] (\i) at (\i,-5) {};
				\foreach \i in {2,...,4,6}
					\draw (\i) -- (\the\numexpr\i+1\relax);
				\draw[dashed] (5) -- (6);
				\node[circle, draw, fill=white, inner sep=1.5pt] (0) at (1,-4.5) {};
				\node[circle, draw, fill=white, inner sep=1.5pt] (1) at (1,-5.5) {};
				\foreach \i in {0,1}
					\draw (\i) -- (2);
        \node[below] at (0) {$\alpha_1$};
        \node[below] at (1) {$\alpha_2$};
        \node[below] at (2) {$\alpha_3$};
        \node[below] at (3) {$\alpha_4$};
        \node[below] at (4) {$\alpha_5$};
        \node[below] at (5) {$\alpha_6$};
        \node[below] at (6) {$\alpha_{n-1}$};
        \node[below] at (7) {$\alpha_n$};
        \node[left] at (1, -5) {$D_n$ : \mbox{\hspace{10pt}}};
        \node[right] at (7) {\mbox{\hspace{10pt}}$(n \ge 4)$};

        % G_2
        \foreach \i in {1,2}
					\node[circle, draw, fill=white, inner sep=1.5pt] (\i) at (\i,-7) {};
				\draw[double distance=3pt, postaction={decorate}, decoration={markings,mark=at position 0.7 with {\arrow{>}}}] (1) -- (2);
				\draw (1) -- (2);
        \node[below] at (1) {$\alpha_1$};
        \node[below] at (2) {$\alpha_2$};
        \node[left] at (1) {$G_2$ : \mbox{\hspace{10pt}}};

        % F_4
        \foreach \i in {1,...,4}
					\node[circle, draw, fill=white, inner sep=1.5pt] (\i) at (\i,-8.5) {};
				\foreach \i in {1,3}
					\draw (\i) -- (\the\numexpr\i+1\relax);
				\draw[double distance=3pt, postaction={decorate}, decoration={markings,mark=at position 0.7 with {\arrow{>}}}] (2) -- (3);
        \node[below] at (1) {$\alpha_1$};
        \node[below] at (2) {$\alpha_2$};
        \node[below] at (3) {$\alpha_3$};
        \node[below] at (4) {$\alpha_4$};
        \node[left] at (1) {$F_4$ : \mbox{\hspace{10pt}}};

        % E_6
        \foreach \i in {1,...,5}
					\node[circle, draw, fill=white, inner sep=1.5pt] (\i) at (\i,-10.5) {};
				\node[circle, draw, fill=white, inner sep=1.5pt] (8) at (3,-9.5) {};
				\foreach \i in {1,...,4}
					\draw (\i) -- (\the\numexpr\i+1\relax);
				\draw (3) -- (8);
        \node[below] at (1) {$\alpha_1$};
        \node[right] at (8) {$\alpha_2$};
        \node[below] at (2) {$\alpha_3$};
        \node[below] at (3) {$\alpha_4$};
        \node[below] at (4) {$\alpha_5$};
        \node[below] at (5) {$\alpha_6$};
        \node[left] at (1) {$E_6$ : \mbox{\hspace{10pt}}};

        % E_7
        \foreach \i in {1,...,6}
					\node[circle, draw, fill=white, inner sep=1.5pt] (\i) at (\i,-12.5) {};
				\node[circle, draw, fill=white, inner sep=1.5pt] (8) at (3,-11.5) {};
				\foreach \i in {1,...,5}
					\draw (\i) -- (\the\numexpr\i+1\relax);
				\draw (3) -- (8);
        \node[below] at (1) {$\alpha_1$};
        \node[right] at (8) {$\alpha_2$};
        \node[below] at (2) {$\alpha_3$};
        \node[below] at (3) {$\alpha_4$};
        \node[below] at (4) {$\alpha_5$};
        \node[below] at (5) {$\alpha_6$};
        \node[below] at (6) {$\alpha_7$};
        \node[left] at (1) {$E_7$ : \mbox{\hspace{10pt}}};

        % E_8
        \foreach \i in {1,...,7}
					\node[circle, draw, fill=white, inner sep=1.5pt] (\i) at (\i,-14.5) {};
				\node[circle, draw, fill=white, inner sep=1.5pt] (8) at (3,-13.5) {};
				\foreach \i in {1,...,6}
					\draw (\i) -- (\the\numexpr\i+1\relax);
				\draw (3) -- (8);
        \node[below] at (1) {$\alpha_1$};
        \node[right] at (8) {$\alpha_2$};
        \node[below] at (2) {$\alpha_3$};
        \node[below] at (3) {$\alpha_4$};
        \node[below] at (4) {$\alpha_5$};
        \node[below] at (5) {$\alpha_6$};
        \node[below] at (6) {$\alpha_7$};
        \node[below] at (7) {$\alpha_8$};
        \node[left] at (1) {$E_8$ : \mbox{\hspace{10pt}}};

        % I_2(m)
        \foreach \i in {1,2}
					\node[circle, draw, fill=white, inner sep=1.5pt] (\i) at (\i,-16) {};
				\draw (1) -- node[midway, above]{$m$} (2);
        \node[below] at (1) {$\alpha_1$};
        \node[below] at (2) {$\alpha_2$};
        \node[left] at (1) {$I_2(m)$ : \mbox{\hspace{10pt}}};
        \node[right] at (7, -16) {$(m \ge 5, m \ne 6)$};

        % H_3
        \foreach \i in {1,...,3}
					\node[circle, draw, fill=white, inner sep=1.5pt] (\i) at (\i,-17.5) {};
				\draw (1) -- node[midway, above]{$5$} (2);
        \draw (2) -- (3);
        \node[below] at (1) {$\alpha_3$};
        \node[below] at (2) {$\alpha_2$};
        \node[below] at (3) {$\alpha_1$};
        \node[left] at (1) {$H_3$ : \mbox{\hspace{10pt}}};

        % H_4
        \foreach \i in {1,...,4}
					\node[circle, draw, fill=white, inner sep=1.5pt] (\i) at (\i,-19) {};
				\draw (1) -- node[midway, above]{$5$} (2);
        \draw (2) -- (3);
        \draw (3) -- (4);
        \node[below] at (1) {$\alpha_4$};
        \node[below] at (2) {$\alpha_3$};
        \node[below] at (3) {$\alpha_2$};
        \node[below] at (4) {$\alpha_1$};
        \node[left] at (1) {$H_4$ : \mbox{\hspace{10pt}}};
      \end{tikzpicture}
    \end{center}
    \caption{既約essential finite reflection groupのCoxeter-Dynkin diagramの分類}
    \label{fig:coxeter_diag_classification}
  \end{figure}
  %\fi

\newpage
\begin{proof}
  まず,これらのdiagramsはfinite reflection groupsに対応することを示す.
  $\{e_1, \ldots, e_r\}$を$\R^r$の標準基底とする.

  \vspace{10pt}
  \noindent
  \underline{$A_n$型}

  \begin{gather}
    V := \set{\sum_{i=1}^{n+1} a_i e_i}{\sum_{i=1}^{n+1} a_i = 0} \subseteq \R^{n+1},\\
    \begin{split}
      \Phi &:= \set{v \in V}{\inner{v}{v} = 2} \cap \bigoplus_{i=1}^{n+1} \Z e_i\\
      &= \set{e_i - e_j}{1 \le i \ne j \le n+1},
    \end{split}\\
    \Delta := \set{\alpha_i := e_i - e_{i+1}}{1 \le i \le n},\\
    W \simeq \mathfrak{S}_{n+1},\ s_{\alpha_i} \mapsto (i\ j).
  \end{gather}
  とすれば,$i \ne j$のとき
  \begin{equation}
    (s_{\alpha_i} s_{\alpha_j}\textrm{の位数})
    = ((i\ i+1)(j\ j+1)\textrm{の位数})
    = \begin{cases}
      2 & (|i - j| \ge 2)\\
      3 & (|i - j| = 1)
    \end{cases}
  \end{equation}
  となるから$A_n$型のdiagramが得られる.

  \vspace{10pt}
  \noindent
  \underline{$B_n$型}

  $|\Phi| = 2n^2$で,長さの$2$乗が$1$のものが$2n$個,$2$のものが$2n(n-1)$個ある.
  \begin{gather}
    V := \R^n,\\
    \begin{split}
      \Phi &:= \set{v \in V}{\inner{v}{v} \in \{1, 2\}} \cap \bigoplus_{i=1}^{n} \Z e_i\\
      &= \set{\pm e_i}{1 \le i \le n} \cup \set{\pm (e_i + e_j)}{1 \le i < j \le n},
    \end{split}\\
    \Delta := \set{\alpha_i := e_i - e_{i+1},\ \alpha_n := e_n}{1 \le i \le n-1},\\
    W \simeq \mathfrak{S}_n \ltimes (\Z/2\Z)^n,\ \textrm{$\mathfrak{S}_n$は$s_{\alpha_i} \mapsto (i\ j)$で,$(\Z/2\Z)^n$は$e_i \mapsto -e_i$で作用.}
  \end{gather}

  \vspace{10pt}
  \noindent
  \underline{$C_n$型}

  $C_n$型は$B_n$型の``双対''である.
  \begin{gather}
    V := \R^n,\\
    \begin{split}
      \Phi &:= \set{v \in V}{\inner{v}{v} \in \{2, 4\}} \cap \bigoplus_{i=1}^{n} \Z e_i\\
      &= \set{\pm 2e_i}{1 \le i \le n} \cup \set{\pm (e_i + e_j)}{1 \le i < j \le n},
    \end{split}\\
    \Delta := \set{\alpha_i := e_i - e_{i+1},\ \alpha_n := 2e_n}{1 \le i \le n-1},\\
    W \simeq \mathfrak{S}_n \ltimes (\Z/2\Z)^n,\ \textrm{作用は$B_n$型と同じ.}
  \end{gather}

  \vspace{10pt}
  \noindent
  \underline{$D_n$型}
  \begin{gather}
    V := \R^n,\\
    \begin{split}
      \Phi &:= \set{v \in V}{\inner{v}{v} = 2} \cap \bigoplus_{i=1}^{n} \Z e_i\\
      &= \set{\pm (e_i + e_j)}{1 \le i < j \le n},
    \end{split}\\
    \Delta := \set{\alpha_i = e_i - e_{i+1},\ \alpha_n = e_{n-1} + e_n}{1 \le i \le n-1},\\
    W \simeq \mathfrak{S}_n \ltimes (\Z/2\Z)^{n-1},\ \textrm{$\mathfrak{S}_n$は$s_{\alpha_i} \mapsto (i\ j)$で,$(\Z/2\Z)^{n-1}$は偶数個の符号変換$e_i \mapsto -e_i$で作用.}
  \end{gather}

  \vspace{10pt}
  \noindent
  \underline{$G_2$型}
  \begin{gather}
    V := \set{\sum_{i=1}^{3} a_i e_i}{\sum_{i=1}^{3} a_i = 0} \subseteq \R^3,\\
    \begin{split}
      \Phi &:= \set{v \in V}{\inner{v}{v} = \{2, 6\}} \cap \bigoplus_{i=1}^{3} \Z e_i\\
      &= \set{\pm(e_i - e_j)}{1 \le i < j \le 3} \cup \set{\pm(2e_i - e_j - e_k)}{\{i, j, k\} = \{1, 2, 3\}},
    \end{split}\\
    \Delta := \{\alpha_1 := e_1 - e_2,\ \alpha_2 = -2e_1 + e_2 + e_3\},\\
    W \simeq \mathcal{D}_6\ (\textrm{位数12の二面体群}).
  \end{gather}

  \vspace{10pt}
  \noindent
  \underline{$F_4$型}

  補助的に格子$\Lambda \subseteq V$を導入する.$|\Phi| = 48$で,長さの$2$乗が$1$のものと$2$のものがそれぞれ$24$個ある.
  \begin{gather}
    V := \R^4,\\
    \Lambda := \bigoplus_{i=1}^{4} \Z e_i + \Z \left( \frac{1}{2} \sum_{i=1}^{4} e_i \right),\\
    \begin{split}
      \Phi &:= \set{v \in \Lambda}{\inner{v}{v} = \{1, 2\}}\\
      &= \set{\pm (e_i + e_j)}{1 \le i < j \le 4} \cup \set{\pm e_i}{1 \le i \le 4} \cup \{\frac{1}{2}(\pm e_1 \pm e_2 \pm e_3 \pm e_4)\},
    \end{split}\\
    \Delta := \{\alpha_1 := e_2-e_3,\ \alpha_2 := e_3-e_4,\ \alpha_3 := e_4,\ \alpha_4 := \frac{1}{2}(e_1-e_2-e_3-e_4)\}.
  \end{gather}

  \vspace{10pt}
  \noindent
  \underline{$E_8$型}

  補助的に格子$\Lambda \subseteq V$を導入する.
  $|\Phi| = 4 \cdot 28 + 2^8 / 2 = 112 + 128 = 240$である.
  \begin{gather}
    V := \R^8,\\
    \Lambda := \set{\sum_{i=1}^{8} c_i e_i}{c_i \in \Z,\ \sum_{i=1}^{8} e_i \in 2\Z} + \Z \left(\frac{1}{2} \sum_{i=1}^{8} e_i\right),\\
    \begin{split}
      \Phi &:= \set{v \in \Lambda}{\inner{v}{v} = 2}\\
      &= \set{\pm (e_i + e_j)}{1 \le i < j \le 8} \cup \set{\frac{1}{2} \sum_{i=1}^{8} \pm e_i}{\textrm{マイナスが偶数個}},
    \end{split}\\
    \Delta := \{\alpha_1 := \frac{1}{2} (e_1 - e_2 - \cdots - e_7 + e_8),\ \alpha_2 := e_1 + e+2,\ \alpha_i = e_{i-1} - e_{i-2}\ (3 \le i \le 8)\}.
  \end{gather}

  \vspace{10pt}
  \noindent
  \underline{$E_7$型}

  Coxeter-Dynkin diagramが$E_8$型の部分グラフであることに注意する.
  $E_8$型の$\alpha_i$を用いる.
  $|\Phi| = 4 \cdot 15 + 2 + 2 \cdot (6 + 20 + 6) = 126$である.
  \begin{gather}
    V := \R\{\alpha_1, \ldots, \alpha_7\} \subseteq \R^8,\\
    \begin{split}
      \Phi &:= \Phi_{E_8} \cap V\\
      &= \set{\pm (e_i + e_j)}{1 \le i < j \le 6} \cup \{\pm (e_7 - e_8)\} \cup \set{\pm \frac{1}{2} \left(e_7 - e_8 + \sum_{i=1}^{6} \pm e_i\right)}{\textrm{$\sum$の中のマイナスは奇数個}},
    \end{split}\\
    \Delta := \{\alpha_1, \ldots, \alpha_7\}.
  \end{gather}

  \vspace{10pt}
  \noindent
  \underline{$E_6$型}

  Coxeter-Dynkin diagramが$E_8$型の部分グラフであることに注意する.
  $E_8$型の$\alpha_i$を用いる.
  \begin{gather}
    V := \R\{\alpha_1, \alpha_6\} \subseteq \R^n,\\
    \begin{split}
      \Phi &:= \Phi_{E_8} \cap V\\
      &= \set{\pm (e_i - e_j)}{1 \le i < j \le 5} \cup \set{\pm \frac{1}{2}\left(e_8 - e_7 - e_6 + \sum_{i=1}^{5} \pm e_i\right)}{\textrm{$\sum$の中のマイナスは奇数個}},
    \end{split}\\
    \Delta := \{\alpha_1, \ldots, \alpha_6\}.
  \end{gather}

\end{proof}

\begin{proposition}
  $W_1, W_2$をessential finite reflection groupsとし,$G_{\Delta_1}, G_{\Delta_2}$をそれぞれ対応するCoxeter-Dynkin diagramとする.
  このとき,$W_1 \times W_2$はessential finite reflection groupであり,そのCoxeter-Dynkin diagramは$G_{\Delta_1} \sqcup G_{\Delta_2}$である.
\end{proposition}

よって,任意のessential finite reflection groupから定まるCoxeter-Dynkin diagramは,図~\ref{fig:coxeter_diag_classification}の組み合わせで表される.



\newpage
\section{$E_8$ lattice}

\begin{definition}
  \label{defi:E8Lattice}
  \lean{E8Lattice}
  \leanok
  \uses{defi:IntegralLattice}
  $E_8$格子とは,integralLatticeであって,even unimodularかつランクが$8$であるもののことである.
\end{definition}

\begin{theorem}
  \label{thm:unique}
  \lean{E8Lattice.unique}
  \leanok
  2つの$E_8$格子$\Lambda_1,\ \Lambda_2$は同型である.
\end{theorem}

\begin{proof}
  sorry.
\end{proof}

\begin{definition}
  \label{defi:M0-M7}
  \lean{E8Lattice.M0}
  \leanok
  $E_8$のCartan行列を$M_0$,それを1行ずつ行基本変形していき(その過程の行列を$M_1, M_2, \ldots, M_6$とする)上三角にしたものを$M_7$とする:
  \begin{gather}
    M_0 :=
    \begin{pmatrix}
      2 & 0 & -1 & 0 & 0 & 0 & 0 & 0 \\
      0 & 2 & 0 & -1 & 0 & 0 & 0 & 0 \\
      -1 & 0 & 2 & -1 & 0 & 0 & 0 & 0 \\
      0 & -1 & -1 & 2 & -1 & 0 & 0 & 0 \\
      0 & 0 & 0 & -1 & 2 & -1 & 0 & 0 \\
      0 & 0 & 0 & 0 & -1 & 2 & -1 & 0 \\
      0 & 0 & 0 & 0 & 0 & -1 & 2 & -1 \\
      0 & 0 & 0 & 0 & 0 & 0 & -1 & 2
    \end{pmatrix},\\
    M_7 :=
    \begin{pmatrix}
      2 & 0 & -1 & 0 & 0 & 0 & 0 & 0 \\
      0 & 2 & 0 & -1 & 0 & 0 & 0 & 0 \\
      0 & 0 & 3/2 & -1 & 0 & 0 & 0 & 0 \\
      0 & 0 & 0 & 5/6 & -1 & 0 & 0 & 0 \\
      0 & 0 & 0 & 0 & 4/5 & -1 & 0 & 0 \\
      0 & 0 & 0 & 0 & 0 & 3/4 & -1 & 0 \\
      0 & 0 & 0 & 0 & 0 & 0 & 2/3 & -1 \\
      0 & 0 & 0 & 0 & 0 & 0 & 0 & 1/2
    \end{pmatrix}.
  \end{gather}
\end{definition}

\begin{lemma}
  \label{lem:M7_upperTrianglar}
  \lean{E8Lattice.M7_upperTrianglar}
  \leanok
  $M_7$は上三角である.
\end{lemma}

\begin{proof}
  \leanok
  \uses{defi:M0-M7}
  略.
\end{proof}

\begin{lemma}
  \label{lem:M7_det}
  \lean{E8Lattice.M7_det}
  \leanok
  $\det M_7 = 1$である.
\end{lemma}

\begin{proof}
  \leanok
  \uses{lem:M7_upperTrianglar, defi:M0-M7}
  補題~\ref{lem:M7_upperTrianglar}より,$M_7$の行列式は対角成分たちの積であるから
  \begin{equation}
    \det M_7
    = 2 \cdot 2 \cdot (3 / 2) \cdot (5 / 6) \cdot (4 / 5) \cdot (3 / 4) \cdot (2 / 3) \cdot (1 / 2)\\
    = 1.
  \end{equation}
\end{proof}

\begin{theorem}
  \label{thm:E8_det}
  \lean{E8Lattice.E8_det}
  \leanok
  $E_8$のCartan行列の行列式は$1$である.
\end{theorem}

\begin{proof}
  \uses{defi:M0-M7, lem:M7_det}
  補題~\ref{lem:M7_det}より
  \begin{equation}
    (\textrm{求める行列式})
    = \det M_0
    = \det M_1
    = \cdots
    = \det M_7
    = 1.
  \end{equation}
\end{proof}

\begin{definition}
  \label{defi:B}
  \lean{E8Lattice.B}
  \leanok
  $B$を$E_8$のCartan行列$C (= M_0) \in \M_8(\Z)$から定まる双線型形式とする:
  \begin{equation}
    B(x, y) := \transpose{x} C y = \inner{x}{Cy} \qquad (\forall x, y \in \Z^8).
  \end{equation}
\end{definition}

\begin{lemma}
  \label{lem:inner_self_calc}
  \lean{E8Lattice.inner_self_calc}
  \leanok
  任意の$x \in \Z^8$に対し,次が成り立つ:
  \begin{equation}
    \begin{split}
      B(x, x) ={}& 2 (x_0^2 + x_1^2 + x_2^2 + x_3^2 + x_4^2 + x_5^2 + x_6^2 + x_7^2\\
      &- (x_0 x_2 + x_1 x_3 + x_2 x_3 + x_3 x_4 + x_4 x_5 + x_5 x_6 + x_6 x_7))
    \end{split}
  \end{equation}
\end{lemma}

\begin{proof}
  \leanok
  \uses{defi:B}
  内積の形にして,あとは具体的に計算:
  \begin{equation}
    B(x, x)
    = \inner{x}{Cx}
    = (\textrm{右辺}).
  \end{equation}
\end{proof}

\begin{lemma}
  \label{lem:inner_self_comp_sq}
  \lean{E8Lattice.inner_self_comp_sq}
  \leanok
  任意の$x \in \Z^8$に対し,平方完成すると次のようになる:
  \begin{equation}
    \begin{split}
      B(x, x) ={}& \left( \sqrt{2} \, x_0 - \sqrt{\frac{1}{2}} \, x_2 \right)^2 + \left( \sqrt{2} \, x_1 - \sqrt{\frac{1}{2}} \, x_3 \right)^2 + \left( \sqrt{\frac{3}{2}} \, x_2 - \sqrt{\frac{2}{3}} \, x_3 \right)^2 \\
      & + \left( \sqrt{\frac{5}{6}} \, x_3 - \sqrt{\frac{6}{5}} \, x_4 \right)^2 + \left( \sqrt{\frac{4}{5}} \, x_4 - \sqrt{\frac{5}{4}} \, x_5 \right)^2 + \left( \sqrt{\frac{3}{4}} \, x_5 - \sqrt{\frac{4}{3}} \, x_6 \right)^2 \\
      & + \left( \sqrt{\frac{2}{3}} \, x_6 - \sqrt{\frac{3}{2}} \, x_7 \right)^2 + \frac{1}{2} \, x_7^2
    \end{split}
  \end{equation}
\end{lemma}

\begin{proof}
  \leanok
  \uses{lem:inner_self_calc}
  左辺に補題~\ref{lem:inner_self_calc}を代入して計算すれば得られる.
\end{proof}

\begin{theorem}
  \label{thm:add_inner}
  \lean{E8Lattice.AddInner}
  \leanok
  $\forall x, y, z \in \Z^8,\ B(x+y, z) = B(x, z) + B(y, z)$.
\end{theorem}

\begin{proof}
  \leanok
  \uses{defi:B}
  計算するだけ.
\end{proof}

\begin{theorem}
  \label{thm:inner_sym}
  \lean{E8Lattice.InnerSym}
  \leanok
  $\forall x, y \in \Z^8,\ B(x, y) = B(y, x)$.
\end{theorem}

\begin{proof}
  \leanok
  \uses{defi:B}
  計算するだけ.
\end{proof}

\begin{theorem}
  \label{thm:inner_self}
  \lean{E8Lattice.InnerSelf}
  \leanok
  $\forall x \in \Z^8,\ B(x, x) \ge 0$.
\end{theorem}

\begin{proof}
  \leanok
  \uses{lem:inner_self_comp_sq}
  補題~\ref{lem:inner_self_comp_sq}より,$B(x, x)$は平方の和で表せるから成り立つ.
\end{proof}

\begin{theorem}
  \label{thm:inner_self_eq_zero}
  \lean{E8Lattice.InnerSelfEqZero}
  \leanok
  $\forall x \in \Z^8,\ B(x, x) = 0 \implies x = 0$.
\end{theorem}

\begin{proof}
  \leanok
  \uses{lem:inner_self_comp_sq}
  補題~\ref{lem:inner_self_comp_sq}より,$B(x, x)$は平方の和で表せ,$= 0$とすると各項が$0$である.
  よって,最後の項に注目すると,$x_7 = 0$である.
  したがって,最後から2番目の項に注目すると,$x_6 = 0$である.
  これを繰り返すと,$x_0 = \cdots = x_7 = 0$を得る.
\end{proof}

\begin{theorem}
  \label{thm:even}
  \lean{E8Lattice.InnerEven}
  \leanok
  $\forall x \in \Z^8,\ 2 \mid \inner{x}{x}_{\Z}$.
\end{theorem}

\begin{proof}
  \leanok
  \uses{lem:inner_self_calc}
  補題~\ref{lem:inner_self_calc}より従う.
\end{proof}

\begin{theorem}
  \label{thm:unimodular}
  %\lean{E8Lattice.unimodular}
  \leanok
  $E_8$格子はunimodularである.
\end{theorem}

\begin{proof}
  sorry.
\end{proof}

\begin{theorem}
  \label{thm:exists_E8}
  \lean{E8Lattice.exists_E8}
  \leanok
  $E_8$格子は存在する.
\end{theorem}

\begin{proof}
  sorry.
\end{proof}

\begin{theorem}
  $E_8$格子$\Lambda$に対し,$\forall n \in \N,\ \#\set{x \in \Lambda}{B(x, x) = n} < \infty$.
\end{theorem}

\begin{proof}
  sorry.
\end{proof}

\begin{lemma}
  \label{lem:card_norm_2}
  \lean{E8Lattice.card_norm_2}
  \leanok
  $E_8$格子$\Lambda$に対し,$\#\set{x \in \Lambda}{B(x, x) = 2} = 240$.
\end{lemma}

\begin{proof}
  sorry.
\end{proof}
